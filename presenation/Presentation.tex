\documentclass{beamer}
\let\Tiny=\tiny %to avoid warnings related to font size and beamer 
\usetheme{Amsterdam}
\usecolortheme{dolphin}
\usepackage{amsmath}
\usepackage{tikz}
\usepackage{verbatim}
\usetikzlibrary{arrows,shapes}
\usepackage{hyperref}
\usepackage[style=verbose,backend=bibtex]{biblatex}
\addbibresource{../sources}
\let\oldfootnotesize\footnotesize
\renewcommand*{\footnotesize}{\oldfootnotesize\tiny}

\title{Parallel network flows}
\subtitle{A follow up seminar to the Parallel Algorithms lecture}
\author{Martin Kalany\inst{1} }
\institute
{
  \inst{1}
  Graduate student in Computer Science\\
  Vienna University of Technology\\
}
\date{\today}

\AtBeginSection[]
{
  \begin{frame}
    \frametitle{Table of Contents}
    \tableofcontents[currentsection]
  \end{frame}
}

\begin{document}
	% Declare layers (for more convenience when drawing graphs)
	\pgfdeclarelayer{background}
	\pgfsetlayers{background,main}

	\frame{\titlepage}
	
	\section{Flow networks}
 	\begin{frame}
	\frametitle{Definitions}
    \begin{block}{Definition: Flow network}
    A \textcolor{red}{flow network}\footnote{\cite{ahuja93}}  is given by $N = (G,s,t,c)$, where
    \begin{itemize}
    		\item $G =(V,A)$ is a directed graph
    		\item $s$ and $t$, $s \neq t$ are the source and terminal node
    		\item $c:A\rightarrow \mathbb{R}_0^{+}$ assigns a capacity $\forall a \in A$
    \end{itemize}
    \end{block}
    \textbf{Assumptions:}
	\begin{itemize}
		\item $G$ is connected
		\item $\nexists P(s,t) \in G$ s.t. $c(P) = \infty$
		\item $G$ is simple, i.e., does not contain loops or parallel arcs
	\end{itemize}
	\end{frame}
  
	\begin{frame}[shrink]
	\frametitle{Network flows}
	\begin{block}{Definition: Flow}
	$f:A \rightarrow \mathbb{R}_0^{+}$ is a \textcolor{red}{flow} if it satisfies:
	\begin{itemize}
		\item \textbf{Capacity constraints:} $f(a) \leq c(a)$ $\forall a \in A$
		\item \textbf{Flow conservation:} 
		$ \sum\limits_{v \in V} f(u,v) =  0 \Leftrightarrow IN(f,v) = OUT(f,v)$ $\forall v \in V \setminus \{s,t\}$
		\item \textbf{Value of a flow:} $\lvert f\rvert = f(V,t)$ 
	\end{itemize}
	\end{block}
	
	\begin{block}{More definitions }
	A flow $f$	
	\begin{itemize}
		\item is a \textcolor{red}{maximum flow} if $\lvert f\rvert \geq \lvert f\rvert$, for any other flow $f$
		\item \textcolor{red}{saturates} an arc a if $f(a) = c(a)$
		\item is \textcolor{red}{maximal flow} if every directed path P(s,t) contains at least one saturated edge
	\end{itemize}
	
	\end{block}
	\end{frame}
	
	\begin{frame}
	\frametitle{An intuitive sequential algorithm}
	
\tikzstyle{vertex}=[circle,fill=black!25,minimum size=20pt,inner sep=0pt]
\tikzstyle{edge} = [draw,thick,->]
\tikzstyle{weight} = [font=\small]
\tikzstyle{selected edge} = [draw,line width=5pt,-,red!75]

%TODO: remove
\tikzstyle{ignored edge} = [draw,line width=5pt,-,black!20]
\tikzstyle{selected vertex} = [vertex, fill=red!24]

\begin{figure}
\begin{tikzpicture}[scale=1.8, auto,swap]
    % draw the vertices
    \foreach \pos/\name in {{(0,2)/s}, {(1.5,3)/v_1}, {(3,3)/v_2},
                            {(1.5,1)/v_3}, {(3,1)/v_4}, {(4.5,2)/t}}
        \node[vertex] (\name) at \pos {$\name$};
    % Connect vertices with edges and draw weights
    \foreach \source/ \dest /\weight in {s/v_1/7, s/v_3/8,v_1/v_3/5,v_1/v_4/9,
                                         v_1/v_2/7, v_2/v_4/15, v_3/v_4/5,
                                         v_2/t/8,
                                         v_4/t/9}
        \path[edge] (\source) -- node[weight] {$\weight$} (\dest);
        \path[edge] (v_4) to[bend right] node[weight] {6}  (v_2);
  
    % Start animating the vertex and edge selection. 
   %\foreach \vertex / \fr in {d/1,a/2,f/3,b/4,e/5,c/6,g/7}
    %    \path<\fr-> node[selected vertex] at (\vertex) {$\vertex$};
    % For convenience we use a background layer to highlight edges
    % This way we don't have to worry about the highlighting covering
    % weight labels. 
    \begin{pgfonlayer}{background}
        \pause
        \foreach \source / \dest in {s/v_1,v_1/v_2}
            \path<+->[selected edge] (\source.center) -- (\dest.center);
    %    \foreach \source / \dest / \fr in {d/b/4,d/e/5,e/f/5,b/c/6,f/g/7}
    %        \path<\fr->[ignored edge] (\source.center) -- (\dest.center);
    \end{pgfonlayer}
\end{tikzpicture}
\end{figure}
	
\end{frame}		
 
\section{Complexity}
\begin{frame}
	\frametitle{This is the second slide}
   \framesubtitle{A bit more information about this}
 		%More content goes here
\end{frame}
 	 
\begin{frame}[allowframebreaks]
\frametitle<presentation>{Literature}    
\printbibliography
\end{frame} 	 
 	 
\end{document}