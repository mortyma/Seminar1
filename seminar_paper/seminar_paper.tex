\documentclass[a4paper,10pt, twocolumn]{article}
\usepackage[utf8]{inputenc}
\usepackage{amsmath}
\usepackage{amssymb}
\usepackage[usenames,dvipsnames]{color}
\usepackage{comment}
\usepackage{listings}
\usepackage{algorithm}
\usepackage{algpseudocode}
\usepackage{hyperref}
\usepackage{graphicx}
\usepackage{epstopdf}
\usepackage{enumitem}
\setlist{nolistsep}


%page boarders
\usepackage[top=3cm, bottom=3cm, left=2cm, right=2cm]{geometry}
\usepackage[style=numeric,backend=bibtex]{biblatex}
\addbibresource{../sources}
%\usepackage[style=mla,babel=hyphen,backend=biber]{biblatex}

\title{Parallel Network Flow Algorithms \\ 
\large A follow up seminar to the Parallel Algorithms lecture\footnote{Jesper Larsson Träff, lecture "Parallel Algorithms", 2012 winter term at TU Wien}}
\author{Martin Kalany, 0825673}

\begin{document}
\maketitle

\section{Abstract}
\label{sec:abstract}

\section{Introduction}
\label{sec:intro}

\section{Network flows}
\label{sec:networkFlows}
A \textbf{flow network}\cite{ahuja93} is given by $N = (G,s,t,c)$, where
\begin{itemize}
	\item $G =(V,E)$ is a directed graph
    \item $s$ and $t$, $s \neq t$ are the source and terminal node
   	\item $c:E\rightarrow \mathbb{R}_0^{+}$ assigns a capacity $\forall a \in E$
   	\item $n=\lvert V\rvert$, $m=\lvert E\rvert$
\end{itemize}

Furthermore, the following assumptions are made:
\begin{itemize}
	\item $G$ is connected
	\item $G$ is simple, i.e., does not contain loops or parallel arcs
	\item $\nexists P(s,t)$ with infinite capacity
\end{itemize}

\medskip
$f:E \rightarrow \mathbb{R}_0^{+}$ is a \textbf{flow} if it satisfies:
\begin{itemize}
	\item \emph{Capacity constraints:} $f(e) \leq c(e)$ $\forall e \in E$
	\item \emph{Flow conservation:} 
	$ \sum\limits_{v \in V} f(u,v) =  0 \Leftrightarrow IN(f,v) = OUT(f,v)$ $\forall v \in V \setminus \{s,t\}$
	\item \emph{Skew symmetry:} $f(v,w) = -f(w,v)$
	\item \emph{Value of a flow:} $\lvert f\rvert = f(V,t)$ 
\end{itemize}

A flow $f$
\begin{itemize}
	\item is a \emph{maximum flow} if $\lvert f\rvert \geq \lvert f'\rvert$, for any other flow $f'$
	\item \emph{saturates} an arc e if $f(e) = c(e)$
	\item is a \emph{maximal (or blocking) flow} if every directed path P(s,t) contains at least one saturated arc
\end{itemize}
\medskip
The \emph{residual capacity} of $e \in V \times V$ w.r.t. a flow $f$ is defined as $c_r(e) = c(e) - f(e)$. $G_r = (V, E_r)$ is the \emph{residual network}, where $E_r = \left\{e \in V \times V \lvert c_r(e) > 0\right\}$. A path $P$ from $s$ to $t$ in $G_r$ is called an \emph{augmenting path} and can be used to increase the flow $f$.





\section{Computitional Complecity}
\label{sec:cc}

\section{Algorithm of Shiloach and Vishkin}
\label{sec:shiloach}

\section{Algorithm of Goldberg and Tarjan}
\label{sec:goldberg}

\section{Further results}
\label{sec:further}

\section{bla}
text \cite{ahuja93} \cite{papa95} \cite{yossi81} \cite{vishkin92} \cite{goldberg89} \cite{goldberg91} \cite{goldberg98} \cite{johnson87} \cite{schieber89} \cite{cherivan89} 

%\pagebreak
\printbibliography
\end{document}
